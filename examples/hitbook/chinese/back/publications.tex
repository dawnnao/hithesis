% !Mode:: "TeX:UTF-8" 
\begin{publication}
\noindent\textbf{发表的相关论文}
\begin{publist}



\item BAO Y, \textbf{TANG Z}, LI H. Compressive-sensing data reconstruction for structural health monitoring: a machine-learning approach [J]. Structural Health Monitoring, 2019, 1475921719844039. (SCI收录,JCR分区Q1, IF=4.870, 对应第二章内容)

\item \textbf{TANG Z}, CHEN Z, BAO Y, et al. Convolutional neural network‐based data anomaly detection method using multiple information for structural health monitoring [J]. Structural Control and Health Monitoring, 2019, 26(1): e2296. (SCI收录,JCR分区Q1, IF=3.740, 对应第二章内容)

\item BAO Y, \textbf{TANG Z}, LI H, et al. Computer vision and deep learning–based data anomaly detection method for structural health monitoring [J]. Structural Health Monitoring, 2019, 18(2): 401-21. (SCI收录,JCR分区Q1, IF=4.870, 对应第三章内容)

\item \textbf{TANG Z}, BAO Y, LI H. Group sparsity-aware convolutional neural network for continuous missing data recovery of structural health monitoring [J]. Structural Health Monitoring, 2020, 1475921720931745. (SCI收录,JCR分区Q1, IF=4.870, 对应第四章内容)

\end{publist}

\noindent\textbf{(二)申请及已获得的专利(无专利时此项不必列出)}
\begin{publist}
\item XXX,XXX. 一种温热外敷药制备方案:中国,88105607.3[P]. 1989-07-26.
\end{publist}

\noindent\textbf{(三)参与的科研项目及获奖情况}
\begin{publist}
\item	XXX,XXX. XX~气体静压轴承技术研究, XX~省自然科学基金项目.课题编号:XXXX.
\item XXX,XXX. XX~静载下预应力混凝土房屋结构设计统一理论. 黑江省科学技术二等奖, 2007.
\end{publist}
%\vfill
%\hangafter=1\hangindent=2em\noindent
%\setlength{\parindent}{2em}
\end{publication}
