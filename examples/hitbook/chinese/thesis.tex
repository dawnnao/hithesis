% !Mode:: "TeX:UTF-8"
%%%%%%%%%%%%%%%%%%%%%%%%%%%%%%%%%%%%%%%%%%%%%%%%%%%%%%%%%%%%%%%%%%%%%%%%%%%%%%%%
%          ,
%      /\^/`\
%     | \/   |                CONGRATULATIONS!
%     | |    |             SPRING IS IN THE AIR!
%     \ \    /                                                _ _
%      '\\//'                                               _{ ' }_
%        ||                     hithesis v3                { `.!.` }
%        ||                                                ',_/Y\_,'
%        ||  ,                   dustincys                   {_,_}
%    |\  ||  |\          Email: yanshuoc@gmail.com             |
%    | | ||  | |            https://yanshuo.name             (\|  /)
%    | | || / /                                               \| //
%    \ \||/ /       https://github.com/dustincys/hithesis      |//
%      `\\//`   \\   \./    \\ /     //    \\./   \\   //   \\ |/ /
%     ^^^^^^^^^^^^^^^^^^^^^^^^^^^^^^^^^^^^^^^^^^^^^^^^^^^^^^^^^^^^^^
%%%%%%%%%%%%%%%%%%%%%%%%%%%%%%%%%%%%%%%%%%%%%%%%%%%%%%%%%%%%%%%%%%%%%%%%%%%%%%%%
\documentclass[fontset=fandol,type=doctor,campus=harbin,glue=false]{hithesisbook}
% 此处选项中不要有空格
%%%%%%%%%%%%%%%%%%%%%%%%%%%%%%%%%%%%%%%%%%%%%%%%%%%%%%%%%%%%%%%%%%%%%%%%%%%%%%%%
% 必填选项
% type=doctor|master|bachelor|postdoc
%%%%%%%%%%%%%%%%%%%%%%%%%%%%%%%%%%%%%%%%%%%%%%%%%%%%%%%%%%%%%%%%%%%%%%%%%%%%%%%%
% 选填选项(选填选项的缺省值已经尽可能满足了大多数需求,除非明确知道自己有什么
% 需求)
% campus=shenzhen|weihai|harbin
%   含义:校区选项,默认harbin
% glue=true|false
%   含义:由于我工规范中要求字体行距在一个闭区间内,这个选项为true表示tex自
%   动选择,为false表示区间内一个最接近版心要求行数的要求的默认值,缺省值为
%   false。
% tocfour=true|false
%   含义:是否添加第四级目录,只对本科文科个别要求四级目录有效,缺省值为
%   false
% fontset=windows|mac|ubuntu|fandol|adobe
%   含义:设置字体,默认情况会自动识别系统,然后设置字体。后两个是开源字体,自行
%   下载安装后设置使用。windows是中易字库,窝工默认常用字体,绝对没毛病。mac和
%   ubuntu 默认分别是华文和思源字库,理论上用什么字库都行。后两种开源字库的安装
%   方法到谷歌上百度一下什么都有了。Linux非ubuntu发行版、非x86架构机器等如何运行
%   可到github issue上讨论。
% tocblank=true|false
%   含义:目录中第一章之前,是否加一行空白。缺省值为true。
% chapterhang=true|false
%   含义:目录的章标题是否悬挂居中,规范中要求章标题少于15字,所以这个选项
%   有无没什么用,除了特殊需求。缺省值为true。
% fulltime=true|false
%   含义:是否全日制,缺省值为true。非全日制如同等学力等,要在cover中设置类
%   型,封面中不同格式
% subtitle=true|false
%   含义:论文题目是否含有副标题,缺省值为false,如果有要在cover中设置副标
%   题内容,封面中显示。
% newgeometry=one|two
%   含义:规范中的自相矛盾之处,版芯是否包含页眉页脚,旧方法是按照包含页眉
%   页脚来设置。该选项是多选选项,如果没有这个选项,缺省值是旧模板的版芯设
%   置方法,如果设置该选项one或two,分别对应两种页眉页码对应版芯线的相对位
%   置。第一种是严格按照规范要求,难看。第二种微调了页眉页码位置,好一点。
% debug=true|false
%   含义:是否显示版芯框和行号,用来调试。默认否。
% openright=true|false
%   含义:博士论文是否要求章节首页必须在奇数页,此选项不在规范要求中,按个
%   人喜好自行决定。 默认否。注意,窝工的默认情况是打印版博士论文要求右翻页
%   ,电子版要求非右翻页且无空白页。如果想DIY(或身不由己DIY)在什么地方右
%   翻页,将这个选项设置为false,然后在目标位置添加`\cleardoublepage`命令即
%   可。
% library=true|false
%   含义:是否为提交到图书馆的电子版。默认否。注意:如果设置成true,那么
%   openright选项将被强制转换为false。
% capcenterlast=true|false
%   含义:图题、表题最后一行是否居中对齐(我工规范要求居中,但不要求居中对
%   齐),此选项不在规范要求中,按个人喜好自行决定。默认否。
% subcapcenterlast=true|false
%   含义:子图图题最后一行是否居中对齐(我工规范要求居中,但不要求居中对齐
%   ),此选项不在规范要求中,按个人喜好自行决定。默认否。
% absupper=true|false
%   含义:中文目录中的英文摘要在中文目录中的大小写样式歧义,在规范中要求首
%   字母大写,在work样例中是全大写。该选项控制是否全大写。默认否。
% bsmainpagenumberline=true|false
%   含义:由于本科生论文官方模板的页码和页眉格式混乱,提供这个选项自定义设
%   置是否在正文中显示页码横线,默认否。
% bsfrontpagenumberline=true|false
%   含义:由于本科生论文官方模板的页码和页眉格式混乱,提供这个选项自定义设
%   置是否在前文中显示页码横线,默认否。
% bsheadrule=true|false
%   含义:由于本科生论文官方模板的页码和页眉格式混乱,提供这个选项自定义设
%   置是否显示页眉横线,默认显示。
% splitbibitem=true|false
%   含义:参考文献每一个条目内能不能断页,应广大刀客要求添加。默认否。
% newtxmath=true|false
%   含义:数学字体是否使用新罗马。默认是。
% chapterbold=true|false
%   含义:本科生章标题在目录和正文中是否加粗
%%%%%%%%%%%%%%%%%%%%%%%%%%%%%%%%%%%%%%%%%%%%%%%%%%%%%%%%%%%%%%%%%%%%%%%%%%%%%%%%
\usepackage{hithesis}
\graphicspath{{figures/}}
\begin{document}
\frontmatter
\input{front/cover} % 封面
\makecover
\input{front/denotation}%物理量名称表,符合规范为主,有要求添加
\tableofcontents %目录
\mainmatter
% % !Mode:: "TeX:UTF-8"

\chapter[哈尔滨工业大学研究生学位论文撰写规范]{哈尔滨工业大学研究生学
  位论文撰写规范}[Harbin Institute of Technology Postgraduate Dissertation Writing Specifications]

研究生学位论文是研究生科学研究工作的全面总结,是描述其研究成果、代表其研究水平的
重要学术文献资料,是申请和授予相应学位的基本依据。学位论文撰写是研究生培养过程的
基本训练之一,必须按照确定的规范认真执行。研究生应严肃认真地撰写学位论文,指导教
师应加强指导,严格把关。

学位论文撰写应实事求是,杜绝造假和抄袭等行为;应符合国家及各专业部门制定的有关标
准,符合汉语语法规范。硕士和博士学位论文,除在字数、理论研究的深度及创造性成果等
方面的要求不同外,撰写规范要求基本一致。人文与社会科学、管理学科可在本撰写规范的
基础上补充制定专业的学术规范。

\section{内容要求}[Content specification]
\subsection{题目}[Title]

题目应以简明的词语,恰当、准确、科学地反映论文最重要的特定内容(一般不超过25字),
应中英文对照。题目通常由名词性短语构成,不能含有标点符号;应尽量避免使用不常用的
缩略词、首字母缩写字、字符、代号和公式等。

如题目内容层次很多,难以简化时,可采用题目和副题目相结合的方法。题目与副题目字数
之和不应超过35字,中文的题目与副题目之间用破折号相连,英文则用冒号相连。副题目起
补充、阐明题目的作用。题目和副题目在整篇学位论文中的不同地方出现时,应保持一致。

\subsection{摘要与关键词}[Abstraction and key words]
\subsubsection{摘要}[Abstraction]

摘要是论文内容的高度概括,应具有独立性和自含性,即不阅读论文的全文,就能通过摘要
了解整个论文的必要信息。摘要应包括本论文研究的目的、理论与实际意义、主要研究内容、
研究方法等,重点突出研究成果和结论。

摘要的内容要完整、客观、准确,应做到不遗漏、不拔高、不添加。摘要应按层次逐段简要
写出,避免将摘要写成目录式的内容介绍。摘要在叙述研究内容、研究方法和主要结论时,
除作者的价值和经验判断可以使用第一人称外,一般使用第三人称,采用“分析了……原因”、
“认为……”、“对……进行了探讨”等记述方法进行描述。避免主观性的评价意见,避免
对背景、目的、意义、概念和一般性(常识性)理论叙述过多。

摘要需采用规范的名词术语(包括地名、机构名和人名)。对个别新术语或无中文译文的术
语,可用外文或在中文译文后加括号注明外文。摘要中不宜使用公式、化学结构式、图表、
非常用的缩写词和非公知公用的符号与术语,不标注引用文献编号。

博士学位论文摘要应包括以下几个方面的内容:

(1)论文的研究背景及目的。简洁准确地交代论文的研究背景与意义、相关领域的研究现
状、论文所针对的关键科学问题,使读者把握论文选题的必要性和重要性。此部分介绍不宜
写得过多,一般不多于400字。

(2)论文的主要研究内容。介绍论文所要解决核心问题开展的主要研究工作以及研究方法
或研究手段,使读者可以了解论文的研究思路、研究方案、研究方法或手段的合理性与先进
性。

(3)论文的主要创新成果。简要阐述论文的新思想、新观点、新技术、新方法、新结论等
主要信息,使读者可以了解论文的创新性。

(4)论文成果的理论和实际意义。客观、简要地介绍论文成果的理论和实际意义,使读者
可以快速获得论文的学术价值。

\subsubsection{关键词}[Keywords]
关键词是供检索用的主题词条。关键词应集中体现论文特色,反映研究成果的内涵,具有语
义性,在论文中有明确的出处,并应尽量采用《汉语主题词表》或各专业主题词表提供的规
范词,应列取3$\sim$6个关键词,按词条的外延层次从大到小排列。

\subsection{目录}[Content]

论文中各章节的顺序排列表,包括论文中全部章、节、条三级标题及其页码。

\subsection{论文正文}[Main body]

论文正文包括绪论、论文主体及结论等部分。

\subsubsection{绪论}
绪论一般作为第1章。绪论应包括:本研究课题的来源、背景及其理论意义与实际意义;国
内外与课题相关研究领域的研究进展及成果、存在的不足或有待深入研究的问题,归纳出将
要开展研究的理论分析框架、研究内容、研究程序和方法。

绪论部分要注意对论文所引用国内外文献的准确标注。绪论的主要研究内容的撰写宜使用将
来时态,切忌将论文目录直接作为研究内容。

\subsubsection{论文主体}
论文主体是学位论文的主要部分,应该结构严谨,层次清晰,重点突出,文字简练、通顺。
论文各章之间应该前后关联,构成一个有机的整体。论文给出的数据必须真实可靠,推理正
确,结论明确,无概念性和科学性错误。对于科学实验、计算机仿真的条件、实验过程、仿
真过程等需加以叙述,避免直接给出结果、曲线和结论。引用他人研究成果或采用他人成说
时,应注明出处,不得将其与本人提出的理论分析混淆在一起。

论文主体各章后应有一节“本章小结”,实验方法或材料等章节可不写“本章小结”。各章
小结是对各章研究内容、方法与成果的简洁准确的总结与概括,也是论文最后结论的依据。

\subsubsection{结论}
结论作为学位论文正文的组成部分,单独排写,不加章标题序号,不标注




% Local Variables:
% TeX-master: "../main"
% TeX-engine: xetex
% End:
% !Mode:: "TeX:UTF-8"
%!TEX root = ../thesis.tex

\chapter{绪论}[Introduction]

这是 \hithesis\ 的示例文档,基本上覆盖了模板中所有格式的设置。建议大家在使用模
板之前,除了阅读《\hithesis\:哈尔滨工业大学学位论文模板》\footnote{即
hithesis.pdf文件},本示例文档也最好能看一看。此示例文档尽量使用到所有的排版格式
,然而对于一些不在我工规范中规定的文档,理论上是由用户自由发挥,这里不给出样例
。需要另行载入的宏包和自定义命令在文件`hithesis.sty'中有示例,这里不列举。

\section{课题背景和研究意义}[Background, objective and significance of the subject]

按《关于出版物上数字用法的试行规定》(1987年1月1日国家语言文字工作委员会等7个单位公布),除习惯用中文数字表示的以外,一般数字均用阿拉伯数字。
(1)公历的世纪、年代、年、月、日和时刻一律用阿拉伯数字,如20世纪,80年代,4时3刻等。年号要用四位数,如1989年,不能用89年。
(2)记数与计算(含正负整数、分数、小数、百分比、约数等)一律用阿拉伯数字,如3/4,4.5%,10个月,500多种等。
(3)一个数值的书写形式要照顾到上下文。不是出现在一组表示科学计量和具有统计意义数字中的一位数可以用汉字,如一个人,六条意见。星期几一律用汉字,如星期六。邻近两个数字并列连用,表示概数,应该用汉字数字,数字间不用顿号隔开,如三五天,七八十种,四十五六岁,一千七八百元等。
(4)数字作为词素构成定型的词、词组、惯用语、缩略语等应当使用汉字。如二倍体,三叶虫,第三世界,“七五”规划,相差十万八千里等。
(5)5位以上的数字,尾数零多的,可改写为以万、亿为单位的数。一般情况下不得以十、百、千、十万、百万、千万、十亿、百亿、千亿作为单位。如~\num{345000000}~公里可改写为3.45亿公里或~\num{34500}~万公里,但不能写为3亿~\num{4500}~万公里或3亿4千5百万公里。
(6)数字的书写不必每格一个数码,一般每两数码占一格,数字间分节不用分位号“,”,凡4位或4位以上的数都从个位起每3位数空半个数码(1/4汉字)。“\num{3000000}”,不要写成“3,000,000”,小数点后的数从小数点起向右按每三位一组分节。一个用阿拉伯数字书写的多位数不能从数字中间转行。
(7)数量的增加或减少要注意下列用词的概念:1)增加为(或增加到)过去的二倍,即过去为一,现在为二;2)增加(或增加了)二倍,即过去为一,现在为三;3)超额80%,即定额为100,现在为180;4)降低到80%,即过去为100,现在为80;5)降低(或降低了)80%,即原来为100,现在为20;6)为原数的1/4,即原数为4,现在为1,或原数为1,现在为0.25。
应特别注意在表达数字减小时,不宜用倍数,而应采用分数。如减少为原来的1/2,1/3等。


\section{结构健康监测异常数据诊断的发展概况}[Overview of the development of anomalous data detection for structural health monitoring]

为了感知和理解复杂工程结构的行为,结构健康监测(Structural Health Monitoring, SHM)系统在航空航天、土木和机械工程等领域得到了广泛的应用\cite{Inman2005,Sohn2003,Li2014}。

在土木工程领域,很多大型基础设施、历史保护建筑已经安装了包括各种类型传感器的监测系统,每天都会产生大量的监测数据\cite{Li2016,Li2015_1,Ni2011,Dalvi2016,Huang2016,Park2016}}。近年来,结构健康监测应用已经从单一的基础设施单元扩展到基础设施网络,并可能在未来覆盖到整个城市的建筑群。可以预见,结构健康监测技术将收集到越来越多的数据。因此需要开发更有效的方法来处理这些结构健康监测大数据。

数据异常是数据密集型系统的一个常见问题。由于传感器的不完善和数据传输质量不高造成的故障,这些问题在结构健康监测系统中非常普遍。这些异常不仅会造成误报,而且会影响结构性能评估。此外,数据预处理(或数据清洗)\cite{Sohn2003}是时间和人力密集型的,因此成本非常高。因此,迫切需要有效的数据清洗算法,使结构健康监测数据更加可靠地进行在线监测和进一步分析。

为了消除测量数据中的异常,可以在所有传感器布设位置设置备用传感器来增加硬件冗余度;当活跃传感器输出表现出异常时,备用传感器将会开始信号采集\cite{Worden2004}。然而,对于拥有大量传感器的结构健康监测系统来说,这将是非常昂贵的。另一种方法是部署自验证(Self-validating, SEVA)传感器\cite{Henry1993,Feng2007}。 与传统传感器相比,SEVA传感器具有更强大的计算单元,可以在线自诊断测量值,并输出多类型数据,如原始测量值、原始不确定度、验证测量值和验证不确定度。目前已经开发了一些SEVA传感器的样机\cite{Feng2007,Shen2012}。SEVA传感器的局限性是在具体场景下,不确定度计算和异常修正的方法依赖人工专家设计开发,这些计算方法根据测量对象和异常模式的不同而有所不同。最近开发的无线智能传感器Xnode\cite{Spencer2017}与SEVA传感器类似,具有自诊断、自校准、自识别和自适应的能力。

在数据传输方面,有线系统通常比无线传输更可靠。在无线数据传输中,数据丢失等数据异常现象比较常见。在无线系统中,传感器网络的高链路质量是保证数据高保真度的关键。链路质量随传输功率、传感器节点之间的距离和运行条件(电磁场环境、温度和湿度影响)而变化。链路质量的增强和度量已被广泛研究\cite{Lin2006,Rondinone2008,Nagayama2010}。

目前,数据缺陷是结构健康监测系统中不可避免的问题。在数据清洗的过程中,对数据进行接受、拒绝或先进行修正,然后再接受到后续的分析过程中。许多数据清洗方法已经被提出。一种技术是诊断离群值,离群值构成了一类频繁的异常值(注意,离群值可能不是错误的数据,实际上是非常有参考价值的数据。例如,结构的密集地震响应与正常运行条件下监测的数据相比是异常的,但却是非常有价值的数据),并使用统计方法诊断和过滤超出置信区间的异常值\cite{Barnett1974,John2002}。为了消除测量噪声,SVD\cite{Yang2003},PCA\cite{Zvokelj2011,Calabrese2012},和小波分析\cite{Katicha2014,Jiang2007}等频谱分析技术被广泛使用。通过信号分解-重构过程,测量噪声得以降低。为了降低时域稀疏但振幅较大的突发噪声,Yang提出了一种基于主成分追踪的去噪方法,该方法可以很好地修正土木结构环境振动响应中的毛噪声数据\cite{Yang2014}。此外,采用自动关联神经网络\cite{Kramer1992}的机器学习技术也被用于降噪和异常检测。自动关联网络的关键特征是一个瓶颈层,强制学习信息压缩。然后可以降低输出数据中的噪声,并利用输入和输出数据之间的残差来检测异常。但是,上述方法主要是针对噪声过滤,或者处理简单的异常情况,如数据丢失和零星的异常值,而无法处理复杂的原位数据和多种数据异常模式。因此,数据清洗过程中往往需要专家的干预,以决定是否应该接受、拒绝或通过特定方法来修正一段数据。这种人工干预的时间、人力成本高昂,且容易出错。在结构健康监测领域,目前鲜有通用的、自动的数据异常检测方法。最近,基于计算机视觉技术和深度学习技术的交叉技术成为本领域的热门研究方向。Tamilselvan等利用深信度网络诊断飞机发动机和电力变压器的故障\cite{Tamilselvan2013}.Wei等利用聚类技术识别了正常运行条件下斜拉桥的电缆张力模式\cite{Wei2015}。 Li等利用多模态深度支持向量分类方法解决了变速箱故障诊断问题\cite{Li2015}。 Ye等利用模式匹配算法对长跨度桥梁进行了基于视觉的动态位移测量\cite{Ye2013}。 Koch等人回顾了基于计算机视觉的民用基础设施的缺陷检测和状态评估\cite{Koch2015}。最近,Cha等人使用卷积神经网络从图像中检测结构裂缝损伤\cite{Cha2017},Yang等人从视频测量中识别出高空间分辨率的振动模式\cite{Yang2017_3}。 Yeum等人开发了一种带有航空视觉传感器平台的图像定位技术,以自动筛选出重点关注的一小部分图像\cite{Yeum2017}。

受实际人工干预(本质上是基于生物视觉的数据采集和基于大脑的决策,具有较高的直觉和综合抽象能力)的启发,我们提出了一种基于计算机视觉和深度学习的数据异常检测方法。所提出的方法可以自动识别结构健康监测系统中的多种数据异常模式,准确率高。

\section{术语排版举例}[Glossaries and index]

术语的定义和使用可以结合索引,灵活使用。
例如,\gtssbp 是一种应用于狄利克雷过程抽样的算法。
下次出现将是另一种格式:\gtssbp 。
还可以切换单复数例如:\gscnas ,下次出现为:\gscnas 。
此处体现了\LaTeX\ 格式内容分离的优势。

\section{引用}[Cite]

\sindex[china]{du!段誉}引文标注遵照GB/T7714-2005,采用顺序编码制。正文中引用文献的标示应置于所引内容最后一个字的右上角,所引文献编号用阿拉伯数字置于方括号“[ ]”中,用小4号字体的上角标。要求:

(1)引用单篇文献时,如“二次铣削\cite{cnproceed}”。

(2)同一处引用多篇文献时,各篇文献的序号在方括号内全部列出,各序号间用“,”,如
遇连续序号,可标注讫序号。如,…形成了多种数学模型\cite{cnarticle,cnproceed}…
注意此处添加\cs{inlinecite}中文空格\inlinecite{cnarticle,cnproceed},可以在cfg文件中修改空格类型。

(3)多次引用同一文献时,在文献序号的“[ ]”后标注引文页码。如,…间质细胞CAMP含量
测定\cite[100-197]{cnarticle}…。…含量测定方法规定
\cite[92]{cnarticle}…。

(4)当提及的参考文献为文中直接说明时,则用小4号字与正文排齐,如“由文献\inlinecite{hithesis2017}可知”

\section{定理和定义等}[Theorem]
\begin{theorem}[\cite{cnproceed}]
宇宙大爆炸是一种爆炸。
\end{theorem}
\begin{definition}[(霍金)]
宇宙大爆炸是一种爆炸。
\end{definition}
\begin{assumption}
宇宙大爆炸是一种爆炸。
\end{assumption}
\begin{lemma}
宇宙大爆炸是一种爆炸。
\end{lemma}
\begin{corollary}
宇宙大爆炸是一种爆炸。
\end{corollary}
\begin{exercise}
宇宙大爆炸是一种爆炸。
\end{exercise}
\begin{problem}[(Albert Einstein)]
宇宙大爆炸是一种爆炸。
\end{problem}
\begin{remark}
宇宙大爆炸是一种爆炸。
\end{remark}
\begin{axiom}[(爱因斯坦)]
宇宙大爆炸是一种爆炸。
\end{axiom}
\begin{conjecture}
宇宙大爆炸是一种爆炸。
\end{conjecture}
\section{图片}[Pictures]
图应有自明性。插图应与文字紧密配合,文图相符,内容正确。选图要力求精练,插图、照
片应完整清晰。机械工程图:采用第一角投影法,严格按照GB4457~GB131-83《机械制图》
标准规定。数据流程图、程序流程图、系统流程图等按GB1526-89标准规定。电气图:图形
符号、文字符号等应符合附录3所列有关标准的规定。流程图:必须采用结构化程序并正确
运用流程框图。对无规定符号的图形应采用该行业的常用画法。坐标图的坐标线均用细实线
,粗细不得超过图中曲线;有数字标注的坐标图,必须注明坐标单位。照片图要求主题和主
要显示部分的轮廓鲜明,便于制版。如用放大或缩小的复制品,必须清晰,反差适中。照片
上应有表示目的物尺寸的标度。引用文献中的图时,除在正文文字中标注参考文献序号以外
,还必须在中、英文表题的右上角标注参考文献序号。

\subsection{博士毕业论文双语题注}[Doctoral picture example]
\begin{figure}[htpb]
\centering
\includegraphics[width = 0.4\textwidth]{golfer}
\bicaption[golfer1]{}{打高尔夫球球的人(博士论文双语题注)}{Fig.$\!$}{The person playing golf (Doctoral thesis)}
\end{figure}

每个图均应有图题(由图序和图名组成),图题不宜有标点符号,图名在图序之后空1个半
角字符排写。图序按章编排,如第1章第一个插图的图号为“图1-1”。图题置于图下,硕士论
文只用中文,博士论文用中、英两种文字,居中书写,中文在上,要求中文用宋体5号字,
英文用Times New Roman 5号字。有图注或其它说明时应置于图题之上。引用图应注明出处
,在图题右上角加引用文献号。图中若有分图时,分图题置于分图之下或图题之下,可以只
用中文书写,分图号用a)、b)等表示。图中各部分说明应采用中文(引用的外文图除外)或
数字符号,各项文字说明置于图题之上(有分图时,置于分图题之上)。图中文字用宋体、
Times New Roman字体,字号尽量采用5号字(当字数较多时可用小5号字,以清晰表达为原
则,但在一个插图内字号要统一)。同一图内使用文字应统一。图表中物理量、符号用斜体
。
\subsection{本硕论文题注}[Other picture example]
\begin{figure}[h]
\centering
\includegraphics[width = 0.4\textwidth]{golfer}
\caption{打高尔夫球的人,硕士论文要求只用汉语}
\end{figure}

\subsection{并排图和子图}[Abreast-picture and Sub-picture example]
\subsubsection{并排图}[Abreast-picture example]

使用并排图时,需要注意对齐方式。默认情况是中部对齐。这里给出中部对齐、顶部对齐
、图片底部对齐三种常见方式。其中,底部对齐方式有一个很巧妙的方式,将长度比较小
的图放在左面即可。

\begin{figure}[htbp]
\centering
\begin{minipage}{0.4\textwidth}
\centering
\includegraphics[width=\textwidth]{golfer}
\bicaption[golfer2]{}{打高尔夫球的人}{Fig.$\!$}{The person playing golf}
\end{minipage}
\centering
\begin{minipage}{0.4\textwidth}
\centering
\includegraphics[width=\textwidth]{golfer}
\bicaption[golfer3]{}{打高尔夫球的人。注意,这里默认居中}{Fig.$\!$}{The person playing golf. Please note that, it is vertically center aligned by default.}
\end{minipage}
\end{figure}

\begin{figure}[htbp]
\centering
\begin{minipage}[t]{0.4\textwidth}
\centering
\includegraphics[width=\textwidth]{golfer}
\bicaption[golfer5]{}{打高尔夫球的人}{Fig.$\!$}{The person playing golf}
\end{minipage}
\centering
\begin{minipage}[t]{0.4\textwidth}
\centering
\includegraphics[width=\textwidth]{golfer}
\bicaption[golfer8]{}{打高尔夫球的人。注意,此图是顶部对齐}{Fig.$\!$}{The person playing golf. Please note that, it is vertically top aligned.}
\end{minipage}
\end{figure}

\begin{figure}[htbp]
\centering
\begin{minipage}[t]{0.4\textwidth}
\centering
\includegraphics[width=\textwidth,height=\textwidth]{golfer}
\bicaption[golfer9]{}{打高尔夫球的人。注意,此图对齐方式是图片底部对齐}{Fig.$\!$}{The person playing golf. Please note that, it is vertically bottom aligned for figure.}
\end{minipage}
\centering
\begin{minipage}[t]{0.4\textwidth}
\centering
\includegraphics[width=\textwidth]{golfer}
\bicaption[golfer6]{}{打高尔夫球的人}{Fig.$\!$}{The person playing golf}
\end{minipage}
\end{figure}

\subsubsection{子图}[Sub-picture example]
注意:子图题注也可以只用中文。规范规定“分图题置于分图之下或图题之下”,但没有给出具体的格式要求。
没有要求的另外一个说法就是“无论什么格式都不对”。
所以只有在一个图中有标注“a),b)”,无法使用\cs{subfigure}的情况下,使用最后一个图例中的格式设置方法,否则不要使用。
为了应对“无论什么格式都不对”,这个子图图题使用“minipage”和“description”环境,宽度,对齐方式可以按照个人喜好自由设置,是否使用双语子图图题也可以自由设置。

\begin{figure}[!h]
\setlength{\subfigcapskip}{-1bp}
\centering
\begin{minipage}{\textwidth}
\centering
\subfigure{\label{golfer41}}\addtocounter{subfigure}{-2}
\subfigure[The person playing golf]{\subfigure[打高尔夫球的人~1]{\includegraphics[width=0.4\textwidth]{golfer}}}
\hspace{2em}
\subfigure{\label{golfer42}}\addtocounter{subfigure}{-2}
\subfigure[The person playing golf]{\subfigure[打高尔夫球的人~2]{\includegraphics[width=0.4\textwidth]{golfer}}}
\end{minipage}
\centering
\begin{minipage}{\textwidth}
\centering
\subfigure{\label{golfer43}}\addtocounter{subfigure}{-2}
\subfigure[The person playing golf]{\subfigure[打高尔夫球的人~3]{\includegraphics[width=0.4\textwidth]{golfer}}}
\hspace{2em}
\subfigure{\label{golfer44}}\addtocounter{subfigure}{-2}
\subfigure[The person playing golf. Here, 'hang indent' and 'center last line' are not stipulated in the regulation.]{\subfigure[打高尔夫球的人~4。注意,规范中没有明确规定要悬挂缩进、最后一行居中。]{\includegraphics[width=0.4\textwidth]{golfer}}}
\end{minipage}
\vspace{0.2em}
\bicaption[golfer4]{}{打高尔夫球的人}{Fig.$\!$}{The person playing gol}
\end{figure}

\begin{figure}[t]
  \centering
  \begin{minipage}{.7\linewidth}
    \setlength{\subfigcapskip}{-1bp}
    \centering
    \begin{minipage}{\textwidth}
      \centering
      \subfigure{\label{golfer45}}\addtocounter{subfigure}{-2}
      \subfigure[The person playing golf]{\subfigure[打高尔夫球的人~1]{\includegraphics[width=0.4\textwidth]{golfer}}}
      \hspace{4em}
      \subfigure{\label{golfer46}}\addtocounter{subfigure}{-2}
      \subfigure[The person playing golf]{\subfigure[打高尔夫球的人~2]{\includegraphics[width=0.4\textwidth]{golfer}}}
    \end{minipage}
    \vskip 0.2em
  \wuhao 注意:这里是中文图注添加位置(我工要求,图注在图题之上)。
    \vspace{0.2em}
\bicaption[golfer47]{}{打高尔夫球的人。注意,此处我工有另外一处要求,子图图题可以位于主图题之下。但由于没有明确说明位于下方具体是什么格式,所以这里不给出举例。}{Fig.$\!$}{The person playing golf. Please note that, although it is appropriate to put subfigures' captions under this caption as stipulated in regulation, but its format is not clearly stated.}
  \end{minipage}
\end{figure}

\begin{figure}[t]
\centering
\begin{tikzpicture}
	\node[anchor=south west,inner sep=0] (image) at (0,0) {\includegraphics[width=0.3\textwidth]{golfer}};
	\begin{scope}[x={(image.south east)},y={(image.north west)}]
		\node at (0.3,0.5) {a)};
		\node at (0.8,0.2) {b)};
	\end{scope}
\end{tikzpicture}
\bicaption[golfer0]{}{打高尔夫球球的人(博士论文双语题注)}{Fig.$\!$}{The person playing golf (Doctoral thesis)}
\vskip -0.4em
 \hspace{2em}
\begin{minipage}[t]{0.3\textwidth}
\wuhao \setlist[description]{font=\normalfont}
	\begin{description}
		\item[a)]子图图题
	\end{description}
 \end{minipage}
 \hspace{2em}
 \begin{minipage}[t]{0.3\textwidth}
\wuhao \setlist[description]{font=\normalfont}
	\begin{description}
		\item[b)]子图图题
		\item[b)]Subfigure caption
	\end{description}
\end{minipage}
\end{figure}


\begin{figure}[!h]
	\centering
	\begin{sideways}
		\begin{minipage}{\textheight}
			\centering
			\fbox{\includegraphics[width=0.2\textwidth]{golfer}}
			\fbox{\includegraphics[width=0.2\textwidth]{golfer}}
			\fbox{\includegraphics[width=0.2\textwidth]{golfer}}
			\fbox{\includegraphics[width=0.2\textwidth]{golfer}}
			\fbox{\includegraphics[width=0.2\textwidth]{golfer}}
			\fbox{\includegraphics[width=0.2\textwidth]{golfer}}
			\fbox{\includegraphics[width=0.2\textwidth]{golfer}}
\bicaption[golfer7]{}{打高尔夫球的人(非规范要求)}{Fig.$\!$}{The person playing golf (Not stated in the regulation)}
		\end{minipage}
	\end{sideways}
\end{figure}

\clearpage

如果不想让图片浮动到下一章节,那么在此处使用\cs{clearpage}命令。

\section{如何做出符合规范的漂亮的图}
关于作图工具在后文\ref{drawtool}中给出一些作图工具的介绍,此处不多言。
此处以R语言和Tikz为例说明如何做出符合规范的图。

\subsection{Tikz作图举例}
使用Tikz作图核心思想是把格式、主题、样式与内容分离,定义在全局中。
注意字体设置可以有两种选择,如何字少,用五号字,字多用小五。
使用Tikz作图不会出现字体问题,字体会自动与正文一致。

\begin{figure}[thb!]
  \centering
      \begin{tikzpicture}[xscale=0.8,yscale=0.3,rotate=90]
        \small
	\draw (-22,6.5) node[refcell]{参考基因组};
	\draw[refline] (-23, 5) -- (27, 5);
	\draw (-22,3.75) node[tscell]{肿瘤样本};
	\draw (-20,3.75) node[tncell]{正常细胞};
	\draw[tnline] (-21, 2.5) -- (27, 2.5);
	\draw (-20,1.25) node[ttcell]{肿瘤细胞};
	\rcell{2}{6};
	\draw[fakeevolve] (4.5, 5.25) -- (4.5, 4.8);
	\ncell{2}{4};
	\draw[evolve] (4.5, 3) .. controls (4.5,2.8) and (-3.5,2.9) ..  (-3.5, 2);
	\draw[evolve] (4.5, 3) .. controls (4.5,2.8) and (11.5,2.9) .. (11.5, 2);
	\tcellone{-6}{1.5};
	\draw (-9, 2) node[ttcell]{1};
	\draw[evolve] (-3.5, 0) .. controls (-3.5,-0.2) and (-12,-0.1) .. (-12, -1.5);
	\draw[evolve] (-3.5, 0) .. controls (-3.5,-0.2) and (1.5,-0.1) .. (1.5, -1.5);
	\tcellthree{7}{1.5};
	\draw (4, 2) node[ttcell]{2};
	\draw[evolve] (11, 0.5) .. controls (11,0.3) and (19,0.4) .. (19, -1.5);
	\tcellfive{-16}{-2};
	\draw (-19, -1.5) node[ttcell]{3};
	\tcelltwo{-1}{-2};
	\draw (-4, -1.5) node[ttcell]{4};
	\tcellfour{12}{-2};
	\draw (9, -1.5) node[ttcell]{5};
      \end{tikzpicture}
  \begin{minipage}{.9\linewidth}
      \vskip 0.2em
      \wuhao 图中,带有箭头的淡蓝色箭头表示肿瘤子种群的进化方向。一般地,从肿瘤组织中取用于进行二代测序的样本中含有一定程度的正常细胞污染,因此肿瘤的样本中含有正常细胞和肿瘤细胞。每一个子种群的基因组的模拟过程是把生殖细胞变异和体细胞变异加入到参考基因组中。
      \vspace{0.6em}
  \end{minipage}
\bicaption[tumor]{}{肿瘤组织中各个子种群的进化示意图}{Fig.$\!$}{The diagram of tumor subpopulation evolution process}
\end{figure}

\subsection{R作图}
R是一种极具有代表性的典型的作图工具,应用广泛。
与Tikz图~\ref{tumor}~不同,R作图分两种情况:(1)可以转换为Tikz码;(2)不可转换为Tikz码。
第一种情况图形简单,图形中不含有很多数据点,使用R语言中的Tikz包即可。
第二种情况是图形复杂,含有海量数据点,这时候不要转成Tikz矢量图,这会使得论文体积巨大。
推荐使用pdf或png非矢量图形。
使用非矢量图形时要注意选择好字号(五号或小五),和字体(宋体、新罗马)然后选择生成图形大小,注意此时在正文中使用\cs{includegraphics}命令导入时,不要像导入矢量图那样控制图形大小,使用图形的原本的
宽度和高度,这样就确保了非矢量图形中的文字与正文一致了。

为了控制\hithesis\ 的大小,此处不给出具体举例,

\section{表格}

表应有自明性。表格不加左、右边线。表的编排建议采用国际通行的三线表。表中文字用宋
体~5~号字。每个表格均应有表题(由表序和表名组成)。表序一般按章编排,如第~1~章第
一个插表的序号为“表~1-1”等。表序与表名之间空一格,表名中不允许使用标点符号,表名
后不加标点。表题置于表上,硕士学位论文只用中文,博士学位论文用中、英文两种文字居
中排写,中文在上,要求中文用宋体~5~号字,英文用新罗马字体~5~号字。表头设计应简单
明了,尽量不用斜线。表头中可采用化学符号或物理量符号。


\subsection{普通表格的绘制方法}[Methods of drawing normal tables]

表格应具有三线表格式,因此需要调用~booktabs~宏包,其标准格式如表~\ref{table1}~所示。
\begin{table}[htbp]
\bicaption[table1]{}{符合研究生院绘图规范的表格}{Table$\!$}{Table in agreement of the standard from graduate school}
\vspace{0.5em}\centering\wuhao
\begin{tabular}{ccccc}
\toprule[1.5pt]
$D$(in) & $P_u$(lbs) & $u_u$(in) & $\beta$ & $G_f$(psi.in)\\
\midrule[1pt]
 5 & 269.8 & 0.000674 & 1.79 & 0.04089\\
10 & 421.0 & 0.001035 & 3.59 & 0.04089\\
20 & 640.2 & 0.001565 & 7.18 & 0.04089\\
\bottomrule[1.5pt]
\end{tabular}
\end{table}
全表如用同一单位,则将单位符号移至表头右上角,加圆括号。表中数据应准确无误,书写
清楚。数字空缺的格内加横线“-”(占~2~个数字宽度)。表内文字或数字上、下或左、右
相同时,采用通栏处理方式,不允许用“〃”、“同上”之类的写法。表内文字说明,起行空一
格、转行顶格、句末不加标点。如某个表需要转页接排,在随后的各页上应重复表的编号。
编号后加“(续表)”,表题可省略。续表应重复表头。

\subsection{长表格的绘制方法}[Methods of drawing long tables]

长表格是当表格在当前页排不下而需要转页接排的情况下所采用的一种表格环境。若长表格
仍按照普通表格的绘制方法来获得,其所使用的\verb|table|浮动环境无法实现表格的换页
接排功能,表格下方过长部分会排在表格第1页的页脚以下。为了能够实现长表格的转页接
排功能,需要调用~longtable~宏包,由于长表格是跨页的文本内容,因此只需要单独的
\verb|longtable|环境,所绘制的长表格的格式如表~\ref{table2}~所示。

注意,长表格双语标题的格式。

\vspace{-1.5bp}
\ltfontsize{\wuhao[1.667]}
\wuhao[1.667]\begin{longtable}{ccc}%
\longbionenumcaption{}{{\wuhao 中国省级行政单位一览
}\label{table3}}{Table$\!$}{}{{\wuhao Overview of the provincial administrative
unit of China}}{-0.5em}{3.15bp}\\
%\caption{\wuhao 中国省级行政单位一览}\\
\toprule[1.5pt] 名称 & 简称 & 省会或首府  \\ \midrule[1pt]
\endfirsthead
\multicolumn{3}{r}{表~\thetable(续表)}\vspace{0.5em}\\
\toprule[1.5pt] 名称 & 简称 & 省会或首府  \\ \midrule[1pt]
\endhead
\bottomrule[1.5pt]
\endfoot
北京市 & 京 & 北京\\
天津市 & 津 & 天津\\
河北省 & 冀 & 石家庄市\\
山西省 & 晋 & 太原市\\
内蒙古自治区 & 蒙 & 呼和浩特市\\
辽宁省 & 辽 & 沈阳市\\
吉林省 & 吉 & 长春市\\
黑龙江省 & 黑 & 哈尔滨市\\
上海市 & 沪/申 & 上海\\
江苏省 & 苏 & 南京市\\
浙江省 & 浙 & 杭州市\\
安徽省 & 皖 & 合肥市\\
福建省 & 闽 & 福州市\\
江西省 & 赣 & 南昌市\\
山东省 & 鲁 & 济南市\\
河南省 & 豫 & 郑州市\\
湖北省 & 鄂 & 武汉市\\
湖南省 & 湘 & 长沙市\\
广东省 & 粤 & 广州市\\
广西壮族自治区 & 桂 & 南宁市\\
海南省 & 琼 & 海口市\\
重庆市 & 渝 & 重庆\\
四川省 & 川/蜀 & 成都市\\
贵州省 & 黔/贵 & 贵阳市\\
云南省 & 云/滇 & 昆明市\\
西藏自治区 & 藏 & 拉萨市\\
陕西省 & 陕/秦 & 西安市\\
甘肃省 & 甘/陇 & 兰州市\\
青海省 & 青 & 西宁市\\
宁夏回族自治区 & 宁 & 银川市\\
新疆维吾尔自治区 & 新 & 乌鲁木齐市\\
香港特别行政区 & 港 & 香港\\
澳门特别行政区 & 澳 & 澳门\\
台湾省 & 台 & 台北市\\
\end{longtable}\normalsize
\vspace{-1em}

此长表格~\ref{table2}~第~2~页的标题“编号(续表)”和表头是通过代码自动添加上去的,无需人工添加,若表格在页面中的竖直位置发生了变化,长表格在第~2~页
及之后各页的标题和表头位置能够始终处于各页的最顶部,也无需人工调整,\LaTeX~系统的这一优点是~word~等软件所无法比拟的。

\subsection{列宽可调表格的绘制方法}[Methods of drawing tables with adjustable-width columns]
论文中能用到列宽可调表格的情况共有两种,一种是当插入的表格某一单元格内容过长以至
于一行放不下的情况,另一种是当对公式中首次出现的物理量符号进行注释的情况,这两种
情况都需要调用~tabularx~宏包。下面将分别对这两种情况下可调表格的绘制方法进行阐述
。
\subsubsection{表格内某单元格内容过长的情况}[The condition when the contents in
some cells of tables are too long]
首先给出这种情况下的一个例子如表~\ref{table3}~所示。
\begin{table}[htbp]
  \centering
\bicaption[table4]{}{最小的三个正整数的英文表示法}{Table$\!$}{The English construction of the smallest three positive integral numbers}\vspace{0.5em}\wuhao
\begin{tabularx}{0.7\textwidth}{llX}
\toprule[1.5pt]
Value & Name & Alternate names, and names for sets of the given size\\\midrule[1pt]
1 & One & ace, single, singleton, unary, unit, unity\\
2 & Two & binary, brace, couple, couplet, distich, deuce, double, doubleton, duad, duality, duet, duo, dyad, pair, snake eyes, span, twain, twosome, yoke\\
3 & Three & deuce-ace, leash, set, tercet, ternary, ternion, terzetto, threesome, tierce, trey, triad, trine, trinity, trio, triplet, troika, hat-trick\\\bottomrule[1.5pt]
\end{tabularx}
\end{table}
tabularx环境共有两个必选参数:第1个参数用来确定表格的总宽度,第2个参数用来确定每
列格式,其中标为X的项表示该列的宽度可调,其宽度值由表格总宽度确定。标为X的列一般
选为单元格内容过长而无法置于一行的列,这样使得该列内容能够根据表格总宽度自动分行
。若列格式中存在不止一个X项,则这些标为X的列的列宽相同,因此,一般不将内容较短的
列设为X。标为X的列均为左对齐,因此其余列一般选为l(左对齐),这样可使得表格美观
,但也可以选为c或r。

\subsubsection{对物理量符号进行注释的情况}[The condition when physical symbols
need to be annotated]

为使得对公式中物理量符号注释的转行与破折号“———”后第一个字对齐,此处最好采用表格
环境。此表格无任何线条,左对齐,且在破折号处对齐,一共有“式中”二字、物理量符号和
注释三列,表格的总宽度可选为文本宽度,因此应该采用\verb|tabularx|环境。由
\verb|tabularx|环境生成的对公式中物理量符号进行注释的公式如式(\ref{eq:1})所示。
\begin{equation}\label{eq:1}
\ddot{\boldsymbol{\rho}}-\frac{\mu}{R_{t}^{3}}\left(3\mathbf{R_{t}}\frac{\mathbf{R_{t}\rho}}{R_{t}^{2}}-\boldsymbol{\rho}\right)=\mathbf{a}
\end{equation}
\begin{tabularx}{\textwidth}{@{}l@{\quad}r@{———}X@{}}
式中& $\boldsymbol{\rho}$ &追踪飞行器与目标飞行器之间的相对位置矢量;\\
&  $\boldsymbol{\ddot{\rho}}$&追踪飞行器与目标飞行器之间的相对加速度;\\
&  $\mathbf{a}$   &推力所产生的加速度;\\
&  $\mathbf{R_t}$ & 目标飞行器在惯性坐标系中的位置矢量;\\
&  $\omega_{t}$ & 目标飞行器的轨道角速度;\\
&  $\mathbf{g}$ & 重力加速度,$=\frac{\mu}{R_{t}^{3}}\left(
3\mathbf{R_{t}}\frac{\mathbf{R_{t}\rho}}{R_{t}^{2}}-\boldsymbol{\rho}\right)=\omega_{t}^{2}\frac{R_{t}}{p}\left(
3\mathbf{R_{t}}\frac{\mathbf{R_{t}\rho}}{R_{t}^{2}}-\boldsymbol{\rho}\right)$,这里~$p$~是目标飞行器的轨道半通径。
\end{tabularx}\vspace{3.15bp}
由此方法生成的注释内容应紧邻待注释公式并置于其下方,因此不能将代码放入
\verb|table|浮动环境中。但此方法不能实现自动转页接排,可能会在当前页剩余空间不够
时,全部移动到下一页而导致当前页出现很大空白。因此在需要转页处理时,还请您手动将
需要转页的代码放入一个新的\verb|tabularx|环境中,将原来的一个\verb|tabularx|环境
拆分为两个\verb|tabularx|环境。

\subsubsection{排版横版表格的举例}[An example of landscape table]

\begin{table}[p]
\centering
\begin{sideways}
\begin{minipage}{\textheight}
\bicaption[table2]{}{不在规范中规定的横版表格}{Table$\!$}{A table style which is not stated in the regulation}
\vspace{0.5em}\centering\wuhao
\begin{tabular}{ccccc}
\toprule[1.5pt]
$D$(in) & $P_u$(lbs) & $u_u$(in) & $\beta$ & $G_f$(psi.in)\\
\midrule[1pt]
 5 & 269.8 & 0.000674 & 1.79 & 0.04089\\
10 & 421.0 & 0.001035 & 3.59 & 0.04089\\
20 & 640.2 & 0.001565 & 7.18 & 0.04089\\
\bottomrule[1.5pt]
\end{tabular}
\end{minipage}
\end{sideways}
\end{table}


\section{公式}
与正常\LaTeX\ 使用方法一致,此处略。关于公式中符号样式的定义在`hithesis.sty'有示
例。

\section{其他杂项}[Miscellaneous]

\subsection{右翻页}[Open right]

对于双面打印的论文,强制使每章的标题页出现右手边为右翻页。
规范中没有明确规定是否是右翻页打印。
模板给出了右翻页选项。
为了应对用户的个人喜好,在希望设置成右翻页的位置之前添加\cs{cleardoublepage}命令即可。

\subsection{算法}[Algorithms]
我工算法有以下几大特点。

(1)算法不在规范中要求。

(2)算法常常被使用(至少计算机学院)。

(3)格式乱,甚至出现了每个实验室的格式要求都不一样。

此处不给出示例,因为没法给,在
\href{https://github.com/dustincys/PlutoThesis}{https://github.com/dustincys/PlutoThesis}
的readme文件中有不同实验室算法要求说明。

\subsection{脚注}[Footnotes]
不在再规范\footnote{规范是指\PGR\ 和\UGR}中要求,模板默认使用清华大学的格式。

\subsection{源码}[Source code]
也不在再规范中要求。如果有需要最好使用minted包,但在编译的时候需要添加“
-shell-escape”选项且安装pygmentize软件,这些不在模板中默认载入,如果需要自行载入
。
\subsection{思源宋体}[Siyuan font]
如果要使用思源字体,需要思源字体的定义文件,此文件请到模板的开发版网址github:
\href{https://github.com/dustincys/hithesis}{https://github.com/dustincys/hithesis}
或者oschia:
\href{https://git.oschina.net/dustincys/hithesis}{https://git.oschina.net/dustincys/hithesis}
处下载。

\subsection{专业绘图工具}[Processional drawing tool]
\label{drawtool}
推荐使用tikz包,使用tikz源码绘图的好处是,图片中的字体与正文中的字体一致。具体如
何使用tikz绘图不属于模板范畴。
tikz适合用来画不需要大量实验数据支撑示意图。但R语言等专业绘图工具具有画出各种、
专业、复杂的数据图。R语言中有tikz包,能自动生成tikz码,这样tikz几乎无所不能。
对于排版有极致追求的小伙伴,可以参考
\href{http://www.texample.net/tikz/resources/}{http://www.texample.net/tikz/resources/}
中所列工具,几乎所有作图软件所作的图形都可转成tikz,然后可以自由的在tikz中修改
图中内容,定义字体等等。实现前文窝工规范中要求的图中字体的一致性的终极目标。


\subsection{术语词汇管理}[Manage glossaries]
推荐使用glossaries包管理术语、缩略语,可以自动生成首次全写,非首次缩写。

\subsection{\TeX\ 源码编辑器}[\TeX editor]
推荐:(1)付费软件Winedt;(2)免费软件kile;(3)vim或emaces或sublime等神级编
译器(需要配置)。

\subsection{\LaTeX\ 排版重要原则}[\LaTeX\ typesetting rules]
格式和内容分离是\LaTeX\ 最大优势,所有多次出现的内容、样式等等都可以定义为简单命
令、环境。这样的好处是方便修改、管理。例如,如果想要把所有的表示向量的符号由粗体
\cs{mathbf}变换到花体\cs{mathcal},只需修改该格式的命令的定义部分,不需要像MS
word那样处处修改。总而言之,使用自定义命令和环境才是正确的使用\LaTeX\ 的方式。

\section{关于捐助}
各位刀客和大侠如用的嗨,要解囊相助,请参照图~\ref{zfb}~中提示操作(二维码被矢量化后之后去
除了头像等冗余无用的部分~)。

\begin{figure}[!h]
\centering\includegraphics[width=0.4\textwidth]{zfb}
\vspace{0.2em}
\bicaption[Donation]{}{捐助,注意此处是子图只用汉语图题的形式,我工规定可以不用
英语图题}{Fig.$\!$}{Donation, please note that it is OK to use Chinese caption
only}
\end{figure}


% Local Variables:
% TeX-master: "../main"
% TeX-engine: xetex
% End:

% !Mode:: "TeX:UTF-8"

\chapter[哈尔滨工业大学研究生学位论文撰写规范]{哈尔滨工业大学研究生学
  位论文\protect\\撰写规范}[Harbin Institute of Technology Postgraduate Dissertation Writing Specifications]

研究生学位论文是研究生科学研究工作的全面总结,是描述其研究成果、代表其研究水平的
重要学术文献资料,是申请和授予相应学位的基本依据。学位论文撰写是研究生培养过程的
基本训练之一,必须按照确定的规范认真执行。研究生应严肃认真地撰写学位论文,指导教
师应加强指导,严格把关。

学位论文撰写应实事求是,杜绝造假和抄袭等行为;应符合国家及各专业部门制定的有关标
准,符合汉语语法规范。硕士和博士学位论文,除在字数、理论研究的深度及创造性成果等
方面的要求不同外,撰写规范要求基本一致。人文与社会科学、管理学科可在本撰写规范的
基础上补充制定专业的学术规范。

\section{内容要求}[Content specification]
\subsection{题目}[Title]

题目应以简明的词语,恰当、准确、科学地反映论文最重要的特定内容(一般不超过25字),
应中英文对照。题目通常由名词性短语构成,不能含有标点符号;应尽量避免使用不常用的
缩略词、首字母缩写字、字符、代号和公式等。

如题目内容层次很多,难以简化时,可采用题目和副题目相结合的方法。题目与副题目字数
之和不应超过35字,中文的题目与副题目之间用破折号相连,英文则用冒号相连。副题目起
补充、阐明题目的作用。题目和副题目在整篇学位论文中的不同地方出现时,应保持一致。

\subsection{摘要与关键词}[Abstraction and key words]
\subsubsection{摘要}[Abstraction]

摘要是论文内容的高度概括,应具有独立性和自含性,即不阅读论文的全文,就能通过摘要
了解整个论文的必要信息。摘要应包括本论文研究的目的、理论与实际意义、主要研究内容、
研究方法等,重点突出研究成果和结论。

摘要的内容要完整、客观、准确,应做到不遗漏、不拔高、不添加。摘要应按层次逐段简要
写出,避免将摘要写成目录式的内容介绍。摘要在叙述研究内容、研究方法和主要结论时,
除作者的价值和经验判断可以使用第一人称外,一般使用第三人称,采用“分析了……原因”、
“认为……”、“对……进行了探讨”等记述方法进行描述。避免主观性的评价意见,避免
对背景、目的、意义、概念和一般性(常识性)理论叙述过多。

摘要需采用规范的名词术语(包括地名、机构名和人名)。对个别新术语或无中文译文的术
语,可用外文或在中文译文后加括号注明外文。摘要中不宜使用公式、化学结构式、图表、
非常用的缩写词和非公知公用的符号与术语,不标注引用文献编号。

博士学位论文摘要应包括以下几个方面的内容:

(1)论文的研究背景及目的。简洁准确地交代论文的研究背景与意义、相关领域的研究现
状、论文所针对的关键科学问题,使读者把握论文选题的必要性和重要性。此部分介绍不宜
写得过多,一般不多于400字。

(2)论文的主要研究内容。介绍论文所要解决核心问题开展的主要研究工作以及研究方法
或研究手段,使读者可以了解论文的研究思路、研究方案、研究方法或手段的合理性与先进
性。

(3)论文的主要创新成果。简要阐述论文的新思想、新观点、新技术、新方法、新结论等
主要信息,使读者可以了解论文的创新性。

(4)论文成果的理论和实际意义。客观、简要地介绍论文成果的理论和实际意义,使读者
可以快速获得论文的学术价值。

\subsubsection{关键词}[Keywords]
关键词是供检索用的主题词条。关键词应集中体现论文特色,反映研究成果的内涵,具有语
义性,在论文中有明确的出处,并应尽量采用《汉语主题词表》或各专业主题词表提供的规
范词,应列取3$\sim$6个关键词,按词条的外延层次从大到小排列。

\subsection{目录}[Content]

论文中各章节的顺序排列表,包括论文中全部章、节、条三级标题及其页码。

\subsection{论文正文}[Main body]

论文正文包括绪论、论文主体及结论等部分。

\subsubsection{绪论}
绪论一般作为第1章。绪论应包括:本研究课题的来源、背景及其理论意义与实际意义;国
内外与课题相关研究领域的研究进展及成果、存在的不足或有待深入研究的问题,归纳出将
要开展研究的理论分析框架、研究内容、研究程序和方法。

绪论部分要注意对论文所引用国内外文献的准确标注。绪论的主要研究内容的撰写宜使用将
来时态,切忌将论文目录直接作为研究内容。

\subsubsection{论文主体}
论文主体是学位论文的主要部分,应该结构严谨,层次清晰,重点突出,文字简练、通顺。
论文各章之间应该前后关联,构成一个有机的整体。论文给出的数据必须真实可靠,推理正
确,结论明确,无概念性和科学性错误。对于科学实验、计算机仿真的条件、实验过程、仿
真过程等需加以叙述,避免直接给出结果、曲线和结论。引用他人研究成果或采用他人成说
时,应注明出处,不得将其与本人提出的理论分析混淆在一起。

论文主体各章后应有一节“本章小结”,实验方法或材料等章节可不写“本章小结”。各章
小结是对各章研究内容、方法与成果的简洁准确的总结与概括,也是论文最后结论的依据。

\subsubsection{结论}
结论作为学位论文正文的组成部分,单独排写,不加章标题序号,不标注




% Local Variables:
% TeX-master: "../main"
% TeX-engine: xetex
% End:
% !Mode:: "TeX:UTF-8"

\chapter[基于计算机视觉和深度学习的结构健康监测异常数据诊断]{基于计算机视觉和深度学习的结构健康监测\protect\\异常数据诊断}[Computer Vision and Deep Learning-based Data Anomaly Detection Method for Structural Health Monitoring]

近年来,结构健康监测(Structural Health Monitoring, SHM)系统在土木基础设施中广泛应用,产生了大量的数据。结构健康监测数据的分析和挖掘已成为土木工程领域的研究热点。然而,由于土木结构的服役恶劣环境,结构健康监测系统测量的数据受到多种异常数据的污染,严重影响了数据分析结果。同时,这也是实现自动实时预警的主要障碍之一,因为很难区分结构破坏造成的异常和与错误数据导致的异常。现有的数据清洗方法主要针对去噪,而错误数据的检测需要领域专业知识,人力、时间成本高昂。受现实世界人工检测过程的启发,本章提出了一种基于计算机视觉和深度学习的数据异常检测方法。方法框架包括两个步骤:通过数据可视化进行数据转换,以及构建和训练深度神经网络(Deep Neural Networks, DNNs)进行异常分类。这个过程模仿了人类的生物视觉和逻辑思考过程。在第一步中,首先将时间序列信号分段可视化为灰度图像,然后转化为图像向量,。在第二步中,由随机选择的手动标记的图像向量组成的训练数据集被输入到一个DNN或一个DNN集群中,通过堆栈式自动编码器(stacked autoencoders)和逐层贪婪训练方法(greedy layer-wise training)对其进行训练。训练后的DNN可用于检测大量未经检查的结构健康监测数据中的潜在异常。采用了中国某座大跨度斜拉桥结构健康监测系统的加速度数据验证了方法可行性和性能。结果表明,本方法可以快速、自动、准确地完成多类别的异常数据诊断。

本章的内容顺序如下:

\section{基于计算机视觉和深度学习的异常数据诊断方法框架}[Framework of the computer vision and deep learning-based data anomaly detection method]

所提出的方法框架包括两个主要步骤,如图1所示:(1)通过数据可视化进行数据转换;(2)对数据异常分类进行DNN训练。该过程模仿了人类的生物视觉和逻辑思维。在数据可视化步骤中,将时间序列信号分成若干部分转化为图像向量,并绘制成灰度图像。在第二步中,由随机选择和手动标记的图像向量组成的训练集被输入到DNNs中,然后通过称为堆栈式自动编码器和贪婪的层明智训练的技术对其进行训练。训练后的DNNs可以检测大量SHM数据中的潜在异常。

\begin{figure}[!h]
\centering\includegraphics[trim=3cm 0cm 3cm 0cm, width=0.9\textwidth]{fig_3-1.pdf}
\vspace{0.2em}
\bicaption[Donation]{}{基于计算机视觉和深度学习的异常数据诊断方法框架}{Fig.$\!$}{Framework of the proposed data anomaly detection method}
\end{figure}

\subsection{数据可视化}[Data visualization]
% \subsubsection{摘要}[Abstraction]

为了像人类专家一样自动检测SHM数据中的多个异常点,第一步是数据可视化。将原始数据分割成一个小时的片段,然后用数字绘制并保存为图像文件,如图1所示。拆分数据可以看作是窗口化数据,相邻两个窗口之间没有重叠。每幅图像的像素分辨率为8位灰度,在保证合理低存储要求的前提下,足以体现数据的图形特征。在这样的分辨率下,每个图像文件的大小小于2 KB,因此每年一个通道的总文件大小约为17 MB。通过像素列的顺序连接组装而成的图像向量是神经网络的输入。数据可视化过程可以看作是一个特征选择过程。

\subsubsection{关键词}[Keywords]
关键词是供检索用的主题词条。关键词应集中体现论文特色,反映研究成果的内涵,具有语
义性,在论文中有明确的出处,并应尽量采用《汉语主题词表》或各专业主题词表提供的规
范词,应列取3$\sim$6个关键词,按词条的外延层次从大到小排列。

\subsection{有监督训练和特征提取}[Supervised training and feature extraction]
\subsubsection{数据标记}[Data labeling]

由于所提出的方法是像人类专家那样通过 "观察 "数字来评价数据质量,因此每幅图像的图形特征是关键点,并选择它们作为分类的标准。然而,SHM系统中的数据异常因结构、传感器类型和放置位置的不同而不同。为了事先获得一定结构的标签知识,如每个异常模式的形式和模式的总数,需要人类专家观察采集数据的图像。

接下来,随机选取标记的训练样本,并手动标记异常模式的索引。这些数据表示为 ,其中为第th个数据片段,是对应的异常模式指数。一般来说,1-5\%的训练配比足以覆盖所有具有代表性的数据异常模式。本文采用单标签分类方法,即当一幅图像出现多个异常特征时,按该图像中所有数据的主要内在特征确定标签。

\subsubsection{深度神经网络构建与训练}[Construction and training of deep neural networks]

人类视觉和计算机视觉的图像获取过程是不同的。在三维空间中,人类是一次性观看图像的全貌,而计算机则是通过连续扫描图像的像素列来 "看 "图像。因此,要向计算机显示图像,就要将图像像素列依次堆叠转化为图像矢量。

为了模仿人类的决策过程,以深刻理解异常的特征,采用了快速有效的Stacked Autoencoder Deep Neural Network.35 DNN是一种具有多个隐藏层的人工神经网络(ANN),它们可以学习输入的高级抽象。20世纪90年代末,由于DNN的训练时间明显偏长,基本被机器学习界所抛弃。2006年,通过引入一种无监督的层-明智的预训练程序,取得了突破性的进展,这使得在短时间内用未标记的数据训练一个深度架构成为可能.36预训练层的堆叠克服了权重调整的消失梯度.35,37。

\begin{figure}[!h]
\centering\includegraphics[width=0.9\textwidth]{fig_3-2}
\vspace{0.2em}
\bicaption[Donation]{}{基于计算机视觉和深度学习的异常数据诊断方法框架}{Fig.$\!$}{Framework of the proposed data anomaly detection method}
\end{figure}

构建了一个双隐藏层叠加的自动编码器神经网络来演示两阶段的训练过程。让表示总共数据片中的第th片源数据,表示采样频率,表示每个图中数据的持续时间。因此,每个的采样点数量为 。 若图像像素阵列的维度为 ,则对应的图像向量的维度为 ,如图3所示。

一般来说,在神经网络中,每个单元都有一个激活值,定义为

其中,是层中单元激活的一般符号;在输入层中, ,而在隐藏层中,.是层中单元的输入值;是层的大小;是与层中单元和层中单元之间的联系相关的权重;是与层中单元相关的偏置,它作为多项式中的常数项,是激活函数。

为了简明扼要的记述,矢量化采用元素化的方式。


自动编码器是一个输入层和输出层大小相同的三层ANN,其中输出值被设置为等于输入。因此,自动编码器的训练是无监督的。本文采用sigmoid函数作为隐藏层和输出层的激活函数。



构建双隐藏层堆叠式自动编码器神经网络的第一阶段是贪婪的层间预训练。将图像向量输入到第一个自动编码器中,设定平均方差(MSE)作为目标函数,采用标度共轭梯度(SCG)算法38来调整权重。训练完成后,提取输入层和隐藏层之间的权重,部署到四层神经网络中,为 。每个训练样本的隐藏层的节点值, ,被输入到第二个自动编码器中。训练完成后,提取输入层和第二自动编码器的隐藏层之间的权值,为 。每个自动编码器的输出值是没有用的。最后,在前期训练中,并将每个训练样本的对应标签输入到softmax分类器中,该分类器能够进行多类分类。softmax分类器中输出层的激活函数定义为

其中,是K维的输出向量;是输入列向量;是某类的预测概率 ,因此, ; 是输入层所有单元与输出层第th个单元之间联系的权重行向量;是权重矩阵


接下来,利用交叉熵作为目标函数来衡量实际标签和预测类之间的误差。它被定义为


其中,为目标函数;为样本数;为指标函数,其中和;为样本的类概率 ,即;和为样本的标签。


代入本训练过程中的符号,softmax分类器的激活函数和目标函数分别给出为

其中表示行中的权重向量,表示训练样本的 。

可以通过使用SCG调谐最小化目标函数来获得。因此,通过一个输入层、两个隐藏层和一个softmax层的有序堆叠,使用与自动编码器相同的激活函数,构建一个四层神经网络。需要注意的是,初始权重设置为 ,而不是随机生成。

第二阶段是微调。组成的训练集被输入到使用SCG调优来训练大网络。使用在预训练过程中获得的初始权重可以保证快速收敛35,37。

训练有素的DNN的部署是灵活的,这取决于SHM系统的结构。大规模民用基础设施的SHM系统通常由多个子网组成,每个子网包含多种传感器类型的许多节点。子系统发生的故障会使同一子网中的节点共享同源数据异常模式,如每个传感器的异常数据模式总数和每个模式的图形特征。此外,某一特定的数据异常模式在不同的传感器和不同的子系统中会有不同的出现。因此,为了获得更好的分类性能,最好在训练前为网络设计一个合适的布局。本文提出了三种DNN的网络布局策略。分别是并行布局、融合布局和多组布局,如图3所示。在并行布局中,每个通道都有一个私有网络来检测其数据异常。这种布局具有较好的本地性能,但无法在传感器之间共享信息。相比之下,融合布局将所有传感器的数据混合起来,训练一个全局网络。多组布局结合了上述两种布局,即同一组中的所有传感器共享一个网络,该网络由其融合数据集进行训练。如果SHM系统的传感器子系统结构信息已知,则可以根据实际的传感器子系统结构将传感器分为不同的组,采用多组布局。如果传感器子系统结构信息未知,则应根据各传感器之间同一异常类别的相似性来选择DNN布局。当不同传感器之间同一类异常差异巨大时,适合采用并行布局;如果不同传感器之间每一类异常相似,则采用融合布局更为实用。

\begin{figure}[!h]
\centering\includegraphics[width=0.9\textwidth]{fig_3-3}
\vspace{0.2em}
\bicaption[Donation]{}{基于计算机视觉和深度学习的异常数据诊断方法框架}{Fig.$\!$}{Framework of the proposed data anomaly detection method}
\end{figure}


\section{大跨度斜拉桥长时距振动监测数据算例验证}[Case study]
\subsection{大跨度斜拉桥结构健康监测系统介绍}[Introduction of the SHM system of a long-span bridge]

作为案例研究,我们考虑中国的一座长跨径斜拉桥(见图4)。该桥主跨1088米,两座边跨各300米,两座306米高的桥塔。自2008年建成以来,该桥的SHM系统包括加速度计、风速计、应变仪、GPS、温度计等。加速度计的信息如图4和表1所示。桥面和桥塔上的加速度计共使用38个通道,其中桥面和桥塔顶部有16个双通道加速度计,桥塔底部有2个三通道加速度计。

在本案例研究中,只考虑SHM系统测得的加速度数据,将加速度数据的异常情况分为六种模式:缺失、微小、离群、平方、趋势和漂移。表2对这六种模式的数据异常特征进行了简要说明。图5为每种模式的典型异常情况。这样的数据异常不仅在加速度数据中常见,在GPS数据、应变数据、风速数据等中也很常见。

\subsection{数据可视化和标记}[Data visualization and labeling]

2012年测得的所有加速数据均绘制在每小时8位灰度图像中,分辨率为100 100像素,共计333792个样本(这里,一个图像就是一个样本)。对于3\%的采样比例,随机选取10014个样本并进行标注,其中50\%生成训练集,另一半作为验证集。作为例子,图6显示了训练集中通道1、2、3各10张图像。请注意,坐标系是不可见的,因为振动响应的持续时间和振幅信息是不需要基于轮廓的分类。图7展示了当样本中存在多异常特征时的5个例子,如2.2节中讨论的那样。

\subsection{DNN设计与训练}[Design and training of DNN]

在这个例子中,桥的SHM系统中传感器的详细信息是未知的。因此,多组DNN布局可能是不可行的。另外,一些数据异常模式在训练数据集中比较少见,容易被遗漏,这将降低训练DNN的数据异常检测精度。采用融合DNN布局,通过共享已标记的数据异常模式来缓解这一问题。

所设计的DNN的架构如图8(a)所示。输入层有10000个节点,因为可视化的图像是一个矢量,其中每个元素都被输入到输入层的一个节点中。输出层有7个节点,分别对应6种模式的数据异常和正常数据。有三个隐藏层,分别有100个、75个、50个节点。每个隐藏层的节点数递减的设计是为了提取较高的抽象特征进行泛化,可以克服随机抽样收集的数据集不平衡导致的过拟合风险。在训练阶段采用了提前停止的方法,图8(b)展示了其性能,表明由交叉熵测量的收敛性在epoch 500时成为最优。图9(a)和9(b)是检查分类结果的混淆矩阵,训练集和验证集的准确率分别为90.7\%和85.8\%。

回收率(最右边一列)是TP数和实际数据集中某个特定模式出现的总次数之比。这个指标评估了分类器从输入事实中获得的可靠性。在训练集和验证集上,"次要 "异常模式的召回率分别只有63.1\%和52.4\%。这种糟糕的表现是因为部分 "次要 "样本被误分类为 "正常",尽管 "正常 "数据的误分类率很低。

精度(混淆矩阵的底行)是预测结果中真阳性(TP)与某类总数的比率。该指标评估了分类器基于输出预测的可靠性。"小 "异常模式的精度高于其召回率,分别达到78.7\%和67.7\%。这一改进是因为来自其他模式的样本被误判为 "次要 "模式的较少。

为了描述的方便,我们用图矩阵来表示特征图。图10说明了所设计的DNN的第一隐藏层中学习的特征。图10中的每个特征图将最大限度地激活第一隐藏层中100个节点中的一个节点,因此公式(5)中的对应输出等于1,有些特征是可以清晰辨认的,如图中 , , , , (表示矩阵中行和列的图),其中的 "趋势 "用 "X "形的交叉线表示。的特点, , , ,学习 "正态 "模式,即水平中间为暗,上下为乱。在 和 中学习 "正方形",边缘清晰。不止一个特征描述了 "次要 "图案,如 , , 和 。"漂移 "与其他层的特征隐性叠加,如在 和 。在 , , ,中的特征会产生明亮的垂直线来代表离群值。最后,在所设计的DNN的第一个隐藏层的特征中没有发现 "缺失 "的模式,虽然这个模式的异常值最多。请注意,"缺失 "模式几乎是空白的纯白色像素,这意味着在第一个隐藏层中没有学习这个模式的特征。不过,"没有特征 "的模式可以通过连续几层的学习步骤中多个特征的叠加来表示。

\subsection{自动异常诊断}[Automatic anomaly detection]


为了测试全训练DNN的数据异常检测能力,采用全年的加速数据。在PC机(CPU:Intel i7-3770,内存:20GB,7200转硬盘)上检测耗时约6小时,与人工检测相比,成本低,省时省力。

图11显示了每个通道中每个数据异常模式的计数。每条的总计数为8784 h,图11中的38个通道大致分为4组:通道1-3、4-12、13-28、29-38。相比较而言,第2组和第4组的数据质量可以接受,7000 h以上的数据都是正常的,大部分异常数据被归为 "缺失 "或 "轻微";第1组和第3组的数据质量较差,6种数据异常模式均占主导地位。计数结果见表3,说明30.08\%的数据为异常数据。而 "轻微 "模式是数据的主要异常形式,占数据总量的10.99\%。

图12显示了2012年的数据异常分布情况。很明显,在空间和时间上出现了几个广泛的集群;这些集群按时间顺序用数字标示。通道1至3构成了集群1,主要由贯穿全年的 "小 "模式组成。注意,集群2包含13~24个通道。如图4所示,这些信道都在大桥主跨的南侧,说明它们很可能在一个子网中,届时这个传感器子系统可能会出现严重的错误。同样,簇3中包含了25-28个通道,这些通道位于大桥南侧边跨,从1月到4月初同时产生了一大块 "方块 "图案。集群4显示,整个SHM系统在4月下旬失效,因为没有一个传感器记录到任何数据。集群5和6由通道5-12中的 "缺失 "图案组成;这些通道位于靠近主跨的北侧。最后,通道13-24均匀地出现故障,产生了与集群2类似的集群7。除了这些群组外,还有零星的异常现象散布在一年中,如图12所示。

数据异常分布与数据异常模式和传感器通道的计数结果分别如图13和图14所示。图13中,"缺失"、"方块"、"趋势"、"漂移 "数据异常模式的分布相对集中,而 "次要"、"离群 "模式在时空上是分散的。图14显示,同组的通道(见图11)不仅总体数据质量相似,而且在异常模式分布和时空趋势上也很相似。以上信息为进一步的数据清洗和分析,以及SHM系统的准确维护提供了指导。此外,所提出的方法还可以大大减少SHM系统中因异常引起的误报次数。

为了验证所提方法的可靠性,对2012年所有图像样本进行人工标注,与所提方法的检测结果进行对比。2012年的实际数据异常分布如图15所示,表明其结果与建议方法得到的图12的检测结果基本一致。实际数据异常的计数结果如表4所示,表明34.09\%的数据存在异常。非常接近于表3所示的检测结果中总数据异常的30.08\%,"小 "模式是数据的主要异常形式,在总数据中占14.92\%。

图16是用于进一步验证所提出方法可靠性的混淆矩阵。对于 "正常 "和 "缺失 "模式,召回率和精度都保持在90\%以上的高值;对于 "小 "模式,42.13\%的 "小 "样本被误判为 "正常",3.43\%的 "正常 "样本被误判为 "小",因此召回率和精度都低于其他模式的;"离群 "模式的召回率和精度适中,分别为74. 0\%和70.6\%;对于 "方块 "模式,实际 "方块 "样本中有76.9\%被正确检测出来,检测结果中93.1\%为真阳性,是可以接受的;"趋势 "模式的召回率和精度较好,分别达到88.6\%和84.5\%,由于 "漂移 "模式量少,精度被其他模式的误分类样本拖累到47.9\%。最后,一年测试数据的总准确率为87.0\%,说明所提出的方法具有良好的SHM数据异常检测能力。

\section{讨论与结果}[Discussion and conclusion]

本文提出了一种基于计算机视觉和深度学习的数据异常检测方法,以自动检测SHM系统中的异常。通过模仿人类专家,首先将SHM时间序列数据转换成图像,供计算机可视化,然后将灰度数字的图像向量作为DNN的训练集。设计好DNN后,采用贪婪的分层训练技术对其进行训练。采用SHM系统对某长跨径斜拉桥的测得的加速度数据来验证所设计和训练的DNN的可行性和准确性。示例中使用的数据包含6种数据异常模式,设计和训练的DNN对数据异常检测结果的全局准确率可以达到87.\%。得到的数据异常分布和传感器侧的异常计数结果,对进一步精确的数据清洗和SHM系统的维护很有帮助。与人工检测方法相比,提出的基于计算机视觉和深度学习的方法效率更高。

所提出的方法为SHM数据预处理提供了一个新的视角,对于SHM系统的自动实时监测和报警,以及基于数据的结构物离线长期性能分析都是必不可少的。虽然本文只关注加速度数据,但也可以应用于其他类型的传感器数据。在今后的工作中,应更多地关注异常点图像的无监督学习表示,以减少人工干预。此外,对于SHM系统测量数据中的并发异常,可以采用多标签分类方法。


% Local Variables:
% TeX-master: "../main"
% TeX-engine: xetex
% End:
\backmatter
\include{back/conclusion}   % 结论
\bibliographystyle{hithesis} %如果没有参考文献时候
\bibliography{reference}
%%%%%%%%%%%%%%%%%%%%%%%%%%%%%%%%%%%%%%%%%%%%%%%%%%%%%%%%%%%%%%%%%%%%%%%%%%%%%%%% 
%-- 注意:以下本硕博、博后书序不一致 --%
%%%%%%%%%%%%%%%%%%%%%%%%%%%%%%%%%%%%%%%%%%%%%%%%%%%%%%%%%%%%%%%%%%%%%%%%%%%%%%%% 
% 硕博书序
%%%%%%%%%%%%%%%%%%%%%%%%%%%%%%%%%%%%%%%%%%%%%%%%%%%%%%%%%%%%%%%%%%%%%%%%%%%%%%%% 
\begin{appendix}%附录
\input{back/appA.tex}
\end{appendix}
% !Mode:: "TeX:UTF-8" 
\begin{publication}
\noindent\textbf{发表的相关论文}
\begin{publist}

\item \textbf{Zhiyi Tang}, Yuequan Bao, and Hui Li. Group sparsity-aware convolutional neural network for missing data recovery of structural health monitoring[J]. Structural Health Monitoring. 2020.

\item	XXX,XXX. Static Oxidation Model of Al-Mg/C Dissipation Thermal Protection Materials[J]. Rare Metal Materials and Engineering, 2010, 39(Suppl. 1): 520-524.(SCI~收录,IDS号为~669JS,IF=0.16)

\item XXX,XXX. 精密超声振动切削单晶铜的计算机仿真研究[J]. 系统仿真学报,2007,19(4):738-741,753.(EI~收录号:20071310514841)

\item XXX,XXX. 局部多孔质气体静压轴向轴承静态特性的数值求解[J]. 摩擦学学报,2007(1):68-72.(EI~收录号:20071510544816)

\item XXX,XXX. 硬脆光学晶体材料超精密切削理论研究综述[J]. 机械工程学报,2003,39(8):15-22.(EI~收录号:2004088028875)

\item XXX,XXX. 基于遗传算法的超精密切削加工表面粗糙度预测模型的参数辨识以及切削参数优化[J]. 机械工程学报,2005,41(11):158-162.(EI~收录号:2006039650087)

\item XXX,XXX. Discrete Sliding Mode Cintrok with Fuzzy Adaptive Reaching Law on 6-PEES Parallel Robot[C]. Intelligent System Design and Applications, Jinan, 2006: 649-652.(EI~收录号:20073210746529)

\end{publist}

\noindent\textbf{(二)申请及已获得的专利(无专利时此项不必列出)}
\begin{publist}
\item XXX,XXX. 一种温热外敷药制备方案:中国,88105607.3[P]. 1989-07-26.
\end{publist}

\noindent\textbf{(三)参与的科研项目及获奖情况}
\begin{publist}
\item	XXX,XXX. XX~气体静压轴承技术研究, XX~省自然科学基金项目.课题编号:XXXX.
\item XXX,XXX. XX~静载下预应力混凝土房屋结构设计统一理论. 黑江省科学技术二等奖, 2007.
\end{publist}
%\vfill
%\hangafter=1\hangindent=2em\noindent
%\setlength{\parindent}{2em}
\end{publication}
    % 所发文章
\include{back/ceindex}    % 索引, 根据自己的情况添加或者不添加,选择自动添加或者手工添加。
\authorization %授权
%\authorization[saomiao.pdf] %添加扫描页的命令,与上互斥
\include{back/acknowledgements} %致谢
% !Mode:: "TeX:UTF-8" 

\begin{resume}
1991~年~3~月~21~日出生于~云南省玉溪市。

2009~年~9~月------2013~年~7~月在~哈尔滨工业大学~土木工程学院~院(系)理论与应用力学专业学习,获得~理学~学士学位。

2014~年~9~月------2016~年~7~月在~哈尔滨工业大学~土木工程学院~院(系)土木工程学科学习,获得~工学~硕士学位。

2016~年~9~月------2021~年~4~月在~哈尔滨工业大学~大学~土木工程~院(系)力学学科学习,获得~工学~博士学位。

获奖情况:如获三好学生、优秀团干部、X~奖学金等(不含科研学术获奖)。

\section{\large Honors}

\vspace{17pt}

\begin{itemize}[label={-}] \itemsep -1pt % Reduce space between items
\item 国家奖学金,教育部 \hfill 2019
\item 工信创新团队一等奖,工信部 \hfill 2019
\item 2018世界交通运输大会(WTC)优秀论文,北京 \hfill 2018
\item 亚太欧智能土木结构暑期学校(APESS)团体第2名(8组),横滨 \hfill 2017
\item 苏交科结构健康监测奖学金 \hfill 2016
\item 哈尔滨工业大学研究生一等奖学金 \hfill 2015
\item 哈尔滨工业大学爱之源奖学金 \hfill 2014
\item 哈尔滨工业大学本科生优秀毕业论文 \hfill 2013
\item 曾宪梓优秀本科生奖学金 \hfill 2010 -- 2013
\item 哈尔滨工业大学栋梁奖学金 \hfill 2010

\end{itemize}

工作经历:

2013~年~9~月------2014~年~7~月参加哈尔滨工业大学研究生支教团赴西藏藏医药大学支教。


% \textbf{( 除全日制硕士生以外,其余学生均应增列此项。个人简历一般应包含教育经历和工作经历。)}
\end{resume}
          % 博士学位论文有个人简介
%%%%%%%%%%%%%%%%%%%%%%%%%%%%%%%%%%%%%%%%%%%%%%%%%%%%%%%%%%%%%%%%%%%%%%%%%%%%%%%% 
% 本科书序为:
%%%%%%%%%%%%%%%%%%%%%%%%%%%%%%%%%%%%%%%%%%%%%%%%%%%%%%%%%%%%%%%%%%%%%%%%%%%%%%%% 
% \authorization %授权
% % \authorization[saomiao.pdf] %添加扫描页的命令,与上互斥
% \include{back/acknowledgements} %致谢
% \begin{appendix}%附录
% \input{back/appendix01}%本科生翻译论文
% \end{appendix}
%%%%%%%%%%%%%%%%%%%%%%%%%%%%%%%%%%%%%%%%%%%%%%%%%%%%%%%%%%%%%%%%%%%%%%%%%%%%%%%% 
% 博后书序
%%%%%%%%%%%%%%%%%%%%%%%%%%%%%%%%%%%%%%%%%%%%%%%%%%%%%%%%%%%%%%%%%%%%%%%%%%%%%%%% 
% \include{back/acknowledgements} %致谢
% \include{back/doctorpublications}    % 所发文章
% % !Mode:: "TeX:UTF-8" 
\begin{publication}
\noindent\textbf{发表的相关论文}
\begin{publist}

\item \textbf{Zhiyi Tang}, Yuequan Bao, and Hui Li. Group sparsity-aware convolutional neural network for missing data recovery of structural health monitoring[J]. Structural Health Monitoring. 2020.

\item	XXX,XXX. Static Oxidation Model of Al-Mg/C Dissipation Thermal Protection Materials[J]. Rare Metal Materials and Engineering, 2010, 39(Suppl. 1): 520-524.(SCI~收录,IDS号为~669JS,IF=0.16)

\item XXX,XXX. 精密超声振动切削单晶铜的计算机仿真研究[J]. 系统仿真学报,2007,19(4):738-741,753.(EI~收录号:20071310514841)

\item XXX,XXX. 局部多孔质气体静压轴向轴承静态特性的数值求解[J]. 摩擦学学报,2007(1):68-72.(EI~收录号:20071510544816)

\item XXX,XXX. 硬脆光学晶体材料超精密切削理论研究综述[J]. 机械工程学报,2003,39(8):15-22.(EI~收录号:2004088028875)

\item XXX,XXX. 基于遗传算法的超精密切削加工表面粗糙度预测模型的参数辨识以及切削参数优化[J]. 机械工程学报,2005,41(11):158-162.(EI~收录号:2006039650087)

\item XXX,XXX. Discrete Sliding Mode Cintrok with Fuzzy Adaptive Reaching Law on 6-PEES Parallel Robot[C]. Intelligent System Design and Applications, Jinan, 2006: 649-652.(EI~收录号:20073210746529)

\end{publist}

\noindent\textbf{(二)申请及已获得的专利(无专利时此项不必列出)}
\begin{publist}
\item XXX,XXX. 一种温热外敷药制备方案:中国,88105607.3[P]. 1989-07-26.
\end{publist}

\noindent\textbf{(三)参与的科研项目及获奖情况}
\begin{publist}
\item	XXX,XXX. XX~气体静压轴承技术研究, XX~省自然科学基金项目.课题编号:XXXX.
\item XXX,XXX. XX~静载下预应力混凝土房屋结构设计统一理论. 黑江省科学技术二等奖, 2007.
\end{publist}
%\vfill
%\hangafter=1\hangindent=2em\noindent
%\setlength{\parindent}{2em}
\end{publication}
    % 所发文章
% % !Mode:: "TeX:UTF-8" 

\begin{resume}
1991~年~3~月~21~日出生于~云南省玉溪市。

2009~年~9~月------2013~年~7~月在~哈尔滨工业大学~土木工程学院~院(系)理论与应用力学专业学习,获得~理学~学士学位。

2014~年~9~月------2016~年~7~月在~哈尔滨工业大学~土木工程学院~院(系)土木工程学科学习,获得~工学~硕士学位。

2016~年~9~月------2021~年~4~月在~哈尔滨工业大学~大学~土木工程~院(系)力学学科学习,获得~工学~博士学位。

获奖情况:如获三好学生、优秀团干部、X~奖学金等(不含科研学术获奖)。

\section{\large Honors}

\vspace{17pt}

\begin{itemize}[label={-}] \itemsep -1pt % Reduce space between items
\item 国家奖学金,教育部 \hfill 2019
\item 工信创新团队一等奖,工信部 \hfill 2019
\item 2018世界交通运输大会(WTC)优秀论文,北京 \hfill 2018
\item 亚太欧智能土木结构暑期学校(APESS)团体第2名(8组),横滨 \hfill 2017
\item 苏交科结构健康监测奖学金 \hfill 2016
\item 哈尔滨工业大学研究生一等奖学金 \hfill 2015
\item 哈尔滨工业大学爱之源奖学金 \hfill 2014
\item 哈尔滨工业大学本科生优秀毕业论文 \hfill 2013
\item 曾宪梓优秀本科生奖学金 \hfill 2010 -- 2013
\item 哈尔滨工业大学栋梁奖学金 \hfill 2010

\end{itemize}

工作经历:

2013~年~9~月------2014~年~7~月参加哈尔滨工业大学研究生支教团赴西藏藏医药大学支教。


% \textbf{( 除全日制硕士生以外,其余学生均应增列此项。个人简历一般应包含教育经历和工作经历。)}
\end{resume}
          % 博士学位论文有个人简介
% \include{back/correspondingaddr} %通信地址
%%%%%%%%%%%%%%%%%%%%%%%%%%%%%%%%%%%%%%%%%%%%%%%%%%%%%%%%%%%%%%%%%%%%%%%%%%%%%%%% 
\end{document}
% Local Variables:
% TeX-engine: xetex
% End:
